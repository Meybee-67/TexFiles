\documentclass[a3paper,12pt]{article}

\usepackage[utf8]{inputenc}
\usepackage[T1]{fontenc}
\usepackage[french]{babel}
\usepackage{lipsum} % pour l'exemple
\usepackage{listings}
\usepackage{geometry}
\geometry{top=1.5cm, bottom=1.5cm, left=0.75cm, right=0.75cm}

% --- Packages nécessaires ---
\usepackage[most]{tcolorbox} % charge les modules nécessaires, dont raster
\tcbuselibrary{raster,skins,breakable} % <- indispensable pour tcbraster
\usepackage{listings}
\usepackage{xcolor}
\usepackage{caption}

\definecolor{turquoise}{RGB}{86, 150, 218}
\definecolor{lightpurple}{RGB}{164,136,241}
\definecolor{orchidpink}{RGB}{234,135,180}
\definecolor{tale}{RGB}{83,174,162}
\definecolor{palegreen}{RGB}{115,176,108}

\makeatletter
\renewcommand{\maketitle}{
  \begin{center}
    \vspace*{-1.5cm} % Ajuste la position verticale du titre
    {\LARGE \@title \par}
    \vspace{0.5em}
    {\large \@author \par}
    \vspace{0.5em}
    {\small \@date \par}
    \vspace{1em}
  \end{center}
}
\makeatother
\title{{\Huge \textbf{Mémento POO en C++}}}
\date{}
\begin{document}
\maketitle
\vspace*{-1.5em}
\noindent
\begin{minipage}[t]{0.5\textwidth}
\vspace{0pt}
\begin{tcolorbox}[title=Principes de la POO, fonttitle=\large, top=0pt, bottom=5pt, boxsep=1pt, colback=white,
  colframe=blue!90!black, 
  colbacktitle=blue!90!black,
  coltitle=white,]
  
\vspace{0.5em}
\subsection*{{\small \underline{Définitions}}}
\begin{itemize}
    \item \textbf{Encapsulation} = regrouper données et fonctions dans une classe\\
    \item \textbf{Abstraction} = simplifier la complexité en masquant les détails\\
    \item \textbf{Héritage} = réutiliser le code d'une autre classe\\
    \item \textbf{Polymorphisme} = utiliser une même interface pour plusieurs formes d’implémentation
\end{itemize}
\end{tcolorbox}
\begin{tcolorbox}[title=Encapsulation, fonttitle=\large, top=0pt, bottom=3pt, boxsep=1pt, colback=white,
  colframe=palegreen, 
  colbacktitle=palegreen,
  coltitle=white,]
  
\vspace{0.5em}
- Consiste à regrouper les données (attributs) et les fonctions (méthodes) au sein d’une même classe\\

- Permet de contrôler l’accès aux attributs,  prévenir les erreurs (en validant les valeurs avant modification) et de masquer les détails d’implémentation
\vspace{-0.5em}
\subsection*{{\small \underline{Niveaux d'accès}}}
\vspace{-0.5em}
- Tous les attributs d'une classe doivent toujours être privés\\

\vspace{-0.1em}
\begin{tabular}{|p{2.5cm}|p{3cm}|p{6.5cm}|}
\hline
\textbf{Mot-clé} & \textbf{Type d’accès} & \textbf{Accessible depuis} \\
\hline
\texttt{private} & Privé & Seulement depuis la classe elle-même \\
\hline
\texttt{protected} & Protégé & Depuis la classe et ses sous-classes \\
\hline
\texttt{public} & Public & Depuis n’importe quelle autre partie du programme \\
\hline
\end{tabular}
\end{tcolorbox}

\begin{tcolorbox}[title=Séparation des fichiers, fonttitle=\large, top=10pt, bottom=6.8pt, boxsep=1pt, colback=white,
  colframe=orchidpink, 
  colbacktitle=orchidpink,
  coltitle=white,]

\end{tcolorbox}
\end{minipage}
\hfill
\begin{minipage}[t]{0.49\textwidth}
\vspace{0pt}
\begin{tcolorbox}[title=Définition d'une classe, fonttitle=\large, top=0pt, bottom=3pt, boxsep=1pt,
colback=white,
  colframe=pink!80!white,, 
  colbacktitle=pink!80!white,
  coltitle=white,]
\vspace{0.5em}
\subsection*{{\small \underline{Définitions}}}
-\textbf{Objet} = instance (un exemplaire) d’une classe\\
- \textbf{Attributs} = variables d'une classe\\
- \textbf{Méthodes} = fonctions d'une classe
\vspace{-0.05em}
\begin{lstlisting}[language=C++, showstringspaces=false, basicstyle=\normalsize , aboveskip=5pt, belowskip=5pt]
class Livre {
private:
    std::string titre;
    std::string auteur;

public:
    // Constructeur
    Livre(std::string t, std::string a)
        : titre(t), auteur(a){ }
    // Methode d'affichage
    void afficher() const {
        std::cout<<titre<<" de "<<auteur<<std::endl;
    }
};
\end{lstlisting}
\end{tcolorbox}

\begin{tcolorbox}[title=Constructeurs et destructeurs, fonttitle=\large, top=0pt, bottom=3pt, boxsep=1pt,colback=white,
  colframe=black!50, 
  colbacktitle=black!50,
  coltitle=white,]
\vspace{0.5em}
\begin{itemize}
    \item \textbf{Constructeur} : initialise un objet
    \item \textbf{Destructeur} : libère la mémoire, même nom précédé de \texttt{\~{}}\\
\end{itemize}
\vspace{-0.5em}
\begin{lstlisting}[language=C++, showstringspaces=false, basicstyle=\normalsize , aboveskip=0pt, belowskip=5pt]
class Livre {
private:
  std::string titre;
public:
  Livre(std::string t) : titre(t) {
      std::cout<<"Livre cree : "<<titre<<std::endl;
    }
  ~Livre() {
      std::cout<<"Livre detruit : "<<titre<<std::endl;
    }
};
\end{lstlisting}
\end{tcolorbox}

\begin{tcolorbox}[title=Séparation des fichiers, fonttitle=\large, top=3pt, bottom=3pt, boxsep=1pt, colback=white,
  colframe=lightpurple, 
  colbacktitle=lightpurple,
  coltitle=white,]

\subsection*{{\small \underline{Principe général}}}
\begin{tabular}{|p{2cm}|p{4.5cm}|p{5cm}|}
\hline
\textbf{Fichier} & \textbf{Contenu} & \textbf{Rôle} \\
\hline
\texttt{Livre.hpp} & Déclarations, prototypes, attributs & Interface publique (structure de la classe) \\
\hline
\texttt{Livre.cpp} & Définitions des méthodes & Implémentation (corps des fonctions) \\
\hline
\texttt{main.cpp} & Fonction \texttt{main()} & Utilisation des classes \\
\hline
\end{tabular}
\subsection*{{\small \underline{Fichier d'en-tête (.hpp)}}}
\begin{lstlisting}[language=C++, showstringspaces=false, basicstyle=\normalsize , aboveskip=2pt, belowskip=0pt]
#ifndef LIVRE_HPP
#define LIVRE_HPP
//def classe Livre
...
\end{lstlisting}
\vspace{-0.6em}
\subsection*{{\small \underline{Fichier source (.cpp)}}}
\begin{lstlisting}[language=C++, showstringspaces=false, basicstyle=\normalsize , aboveskip=2pt, belowskip=0pt]
#include "Livre.hpp"

// Constructeur
Livre::Livre(std::string t, std::string a)
    : titre(t), auteur(a) {}

// Methode d'affichage
void Livre::afficher() const {
    std::cout<<titre<<" de "<<auteur<<std::endl;
}
\end{lstlisting}
\vspace{-0.45em}
\subsection*{{\small \underline{Programme principal}}}
\begin{lstlisting}[language=C++, showstringspaces=false, basicstyle=\normalsize , aboveskip=2pt, belowskip=0pt]
#include "Livre.hpp"

int main() {
    Livre l("1984", "George Orwell");
    l.afficher();
\end{lstlisting}
\end{tcolorbox}
\end{minipage}
\end{document}

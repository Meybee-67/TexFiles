\documentclass[a3paper,12pt]{article}

\usepackage[utf8]{inputenc}
\usepackage[T1]{fontenc}
\usepackage[french]{babel}
\usepackage{lipsum} % pour l'exemple
\usepackage{listings}
\usepackage{geometry}
\geometry{top=1.5cm, bottom=1.5cm, left=0.75cm, right=0.75cm}

% --- Packages nécessaires ---
\usepackage[most]{tcolorbox} % charge les modules nécessaires, dont raster
\tcbuselibrary{raster,skins,breakable} % <- indispensable pour tcbraster
\usepackage{listings}
\usepackage{xcolor}
\usepackage{caption}

\definecolor{turquoise}{RGB}{86, 150, 218}
\definecolor{lightpurple}{RGB}{164,136,241}
\definecolor{orchidpink}{RGB}{234,135,180}
\definecolor{tale}{RGB}{83,174,162}
\definecolor{palegreen}{RGB}{115,176,108}

\makeatletter
\renewcommand{\maketitle}{
  \begin{center}
    \vspace*{-1.5cm} % Ajuste la position verticale du titre
    {\LARGE \@title \par}
    \vspace{0.5em}
    {\large \@author \par}
    \vspace{0.5em}
    {\small \@date \par}
    \vspace{1em}
  \end{center}
}
\makeatother
\title{{\Huge \textbf{Mémento langage C}}}
\date{}
\begin{document}

\maketitle
\vspace*{-1.5em}
\noindent
\begin{minipage}[t]{0.5\textwidth}
\vspace{0pt}
\begin{tcolorbox}[title=Types de base, fonttitle=\large, top=0pt, bottom=5pt, boxsep=1pt, colback=white,
  colframe=blue!90!black, 
  colbacktitle=blue!90!black,
  coltitle=white,]
  
\vspace{0.5em}
\subsection*{{\small \underline{Primitifs}}}
\begin{tabular}{|l|l|l|}
\hline
\textbf{Type} & \textbf{Description} & \textbf{Exemple}\\
\hline
int & Entier & \lstinline|int a =5;| \\
\hline
float & Nombre réel en virgule flottante & \lstinline|float x = 3.14;|\\
\hline
double & Nombre réel (~15 chiffres significatifs) & \lstinline|double y = 2.718;|\\
\hline
char & Caractère & \lstinline|char c = 'A';|\\
\hline
\end{tabular}

\subsection*{{\small \underline{Qualificateurs}}}
\begin{itemize}
\item short : entier court (\lstinline|short int|)
\item long : entier long (\lstinline|long int|)
\item signed : entier signé (valeurs positives et négatives)
\item unsigned : entier non signé (valeurs positives uniquement)
\end{itemize}
\end{tcolorbox}

\begin{tcolorbox}[title=Variables, fonttitle=\large, top=0pt, bottom=3pt, boxsep=1pt, colback=white,
  colframe=palegreen, 
  colbacktitle=palegreen,
  coltitle=white,]
  
\vspace{0.5em}
\subsection*{{\small \underline{Afficher le contenu d'une variable }}}
\vspace{-0.5em}
\begin{lstlisting}[language=C,showstringspaces=false]
int nombreDeVies=4;
printf("Il vous reste %d vies",nombreDeVies);
\end{lstlisting}

\begin{tabular}{|l|p{4cm}|}
\hline
\textbf{Format} & \textbf{Type attendu}\\
\hline
$\%$d & \lstinline|int|\\
\hline
$\%$f & \lstinline|float|\\
\hline
$\%$c & \lstinline|char|\\
\hline
$\%$s & \lstinline|char[ ]| ou \lstinline|char*|\\
\hline
$\%$u & unsigned int\\
\hline
$\%$ld & \lstinline|long|\\
\hline
$\%$x & \lstinline|hex|\\
\hline
\end{tabular}
\vspace{-0.5em}
\subsection*{{\small \underline{Lire le contenu d'une variable }}}
\vspace{-0.5em}
\begin{lstlisting}[language=C,showstringspaces=false]
float rayon, aire;
printf("Quelle est la longueur du rayon ?");
scanf("%f", &rayon);
aire = 3.14*rayon*rayon;
printf("Aire du cercle: %f",aire);
\end{lstlisting}
\end{tcolorbox}

\begin{tcolorbox}[title=Nombres complexes, fonttitle=\large, top=0pt, bottom=5pt, boxsep=1pt, colback=white,
  colframe=turquoise, 
  colbacktitle=turquoise,
  coltitle=white,]
  
\vspace{0.5em}
\subsection*{{\small \underline{Écriture }}}
\begin{itemize}
\item Inclusion bibliothèque : \lstinline|#include <complex.h>|

\item Déclaration: \lstinline|double complex z = 1.0+2.0*I;|

\end{itemize}

\vspace{-1.1em}
\subsection*{{\small \underline{Fonctions utiles}}}

\begin{tabular}{|l|l|}
\hline
\begin{minipage}[t]{7.5cm}
\vspace{-\baselineskip}
\begin{lstlisting}[language=C, basicstyle=\normalsize , aboveskip=5pt, belowskip=5pt]
double real = creal(z)
\end{lstlisting}
\vspace{-\baselineskip}
\end{minipage}
& Partie réelle de z\\
\hline
\begin{minipage}[t]{6.5cm}
\vspace{-\baselineskip}
\begin{lstlisting}[language=C, basicstyle=\normalsize , aboveskip=5pt, belowskip=5pt]
double real = cimag(z)
\end{lstlisting}
\vspace{-\baselineskip}
\end{minipage}
& Partie imaginaire de z\\
\hline
\begin{minipage}[t]{6.5cm}
\vspace{-\baselineskip}
\begin{lstlisting}[language=C, basicstyle=\normalsize , aboveskip=5pt, belowskip=5pt]
absval = cabs(z);
\end{lstlisting}
\vspace{-\baselineskip}
\end{minipage}
& Module de z\\
\hline
\begin{minipage}[t]{6.5cm}
\vspace{-\baselineskip}
\begin{lstlisting}[language=C, basicstyle=\normalsize , aboveskip=5pt, belowskip=5pt]
double complex conjz = conj(z);
\end{lstlisting}
\vspace{-\baselineskip}
\end{minipage}
& Conjugué de z\\
\hline
\end{tabular}
\vspace{0.05em}
\end{tcolorbox}


\begin{tcolorbox}[title=Fonctions, fonttitle=\large, top=10pt, bottom=6.8pt, boxsep=1pt, colback=white,
  colframe=orchidpink, 
  colbacktitle=orchidpink,
  coltitle=white,]
  \vspace{-0.25em}
\subsection*{{\small \underline{Structure}}}
\vspace{-0.5em}
\begin{lstlisting}[language=C, basicstyle=\normalsize , aboveskip=0pt, belowskip=0pt]
type_retour nom_fonction(type1 param1, ...) {
    // instructions
    return valeur;// si type_retour!= void
}
\end{lstlisting}
\vspace{-0.75em}
\subsection*{{\small \underline{Appel de la fonction}}}
- La définition de la fonction se fait avant le \textit{main}
\begin{lstlisting}[language=C, basicstyle=\normalsize , aboveskip=5pt, belowskip=0pt]
int addition(int a, int b) {
    return a + b;
}
int main() {
    int resultat = addition(5, 3);
    printf("%d", resultat);
    return 0;
}
\end{lstlisting}
\vspace{-1.15em}
\subsection*{{\small \underline{Fonction récursive}}}
\begin{lstlisting}[language=C, basicstyle=\normalsize , aboveskip=0pt, belowskip=0pt]
int factorielle(int n) {
    if (n <= 1) return 1;
    return n * factorielle(n - 1);
}
\end{lstlisting}
\end{tcolorbox}
\end{minipage}
\hfill
\begin{minipage}[t]{0.49\textwidth}
\vspace{0pt}
\begin{tcolorbox}[title=Opérations de base, fonttitle=\large, top=0pt, bottom=3pt, boxsep=1pt,
colback=white,
  colframe=pink!80!white,, 
  colbacktitle=pink!80!white,
  coltitle=white,]
\vspace{0.5em}
\subsection*{{\small \underline{Opérateurs arithmétiques}}}

\begin{tabular}{|c|c|p{7cm}|}
\hline
\textbf{Symbole} & \textbf{Opération} & \textbf{Exemple} \\
\hline
+ & Addition & 
\begin{minipage}[t]{6.5cm}
\vspace{-\baselineskip}
\begin{lstlisting}[language=C, basicstyle=\normalsize , aboveskip=5pt, belowskip=20pt]
int a=10, b=4;
int somme = a+b;//14
\end{lstlisting}
\vspace{-\baselineskip}
\end{minipage} \\
\hline
- & Soustraction & 
\begin{minipage}[t]{6.5cm}
\vspace{-\baselineskip}
\begin{lstlisting}[language=C, basicstyle=\normalsize , aboveskip=5pt, belowskip=20pt]
int a=10, b=4;
int difference = a-b;//6
\end{lstlisting}
\vspace{-\baselineskip}
\end{minipage} \\
\hline
* & Multiplication & 
\begin{minipage}[t]{6.5cm}
\vspace{-\baselineskip}
\begin{lstlisting}[language=C, basicstyle=\normalsize , aboveskip=5pt, belowskip=20pt]
int a=10, b=4;
int produit = a*b;//40
\end{lstlisting}
\vspace{-\baselineskip}
\end{minipage} \\
\hline
/ & Division & 
\begin{minipage}[t]{6.5cm}
\vspace{-\baselineskip}
\begin{lstlisting}[language=C, basicstyle=\normalsize , aboveskip=5pt, belowskip=20pt]
int a=10, b=4;
float c=10.0, b=4.0;
int quotient_1 = a/b;//2
float quotient_2 = c/b;//2.5
\end{lstlisting}
\vspace{-\baselineskip}
\end{minipage} \\
\hline
$\%$ & Modulo & 
\begin{minipage}[t]{6.5cm}
\vspace{-\baselineskip}
\begin{lstlisting}[language=C, basicstyle=\normalsize , aboveskip=5pt, belowskip=20pt]
int a=10, b=4;
int reste = a % b;//2
\end{lstlisting}
\vspace{-\baselineskip}
\end{minipage} \\
\hline
\end{tabular}
\end{tcolorbox}

\begin{tcolorbox}[title=Conditions, fonttitle=\large, top=0pt, bottom=3pt, boxsep=1pt,colback=white,
  colframe=black!50, 
  colbacktitle=black!50,
  coltitle=white,]
\vspace{0.5em}
\subsection*{{\small \underline{Condition simple}}}
\vspace{-0.5em}
\begin{lstlisting}[language=C,showstringspaces=false]
if(nombreDeVies==0){
  printf("Game Over");}
else{
  printf("Il vous reste %d vies", nombreDeVies);}
\end{lstlisting}
\vspace{-1em}
\subsection*{{\small \underline{Plusieurs conditions à la fois}}}
\begin{tabular}{|l|p{4cm}|}
\hline
\textbf{Symbole} & \textbf{Signification}\\
\hline
$\&\&$ & ET logique\\
\hline
|| & OU logique\\
\hline
! & NON logique\\
\hline
\end{tabular}
\begin{lstlisting}[language=C,showstringspaces=false]
if(b<a && b>4){
  c=0;}
else{
  c=7*a+b;}
printf("c = %d",c);
\end{lstlisting}
\end{tcolorbox}

\begin{tcolorbox}[title=Boucles, fonttitle=\large, top=3pt, bottom=3pt, boxsep=1pt, colback=white,
  colframe=lightpurple, 
  colbacktitle=lightpurple,
  coltitle=white,]
  
- L’instruction  \textbf{break}  permet d'arrêter l’exécution d’une boucle\\

- L’instruction  \textbf{continue}  permet de passer directement à l’itération suivante sans exécuter les lignes de l’itération courante
\subsection*{{\small \underline{Boucle while}}}
- Tant que la condition est vraie, on répète les instructions entre\\ accolades
\begin{lstlisting}[language=C,showstringspaces=false]
int compteur = 0;
while (compteur < 10)
{
    printf("Bienvenue sur OpenClassrooms !\n");
    compteur++;
}
\end{lstlisting}
\vspace{-1em}
\subsection*{{\small \underline{Boucle do...while}}}
- Cette boucle s'exécutera toujours au moins une fois car le test se fait à la fin

\begin{lstlisting}[language=C,showstringspaces=false]
int compteur = 0;
do
{
    printf("Bienvenue sur OpenClassrooms !\n");
    compteur++;
} while (compteur < 10);
\end{lstlisting}
\vspace{-1em}
\subsection*{{\small \underline{Boucle for}}}
\vspace{-1em}
\begin{lstlisting}[language=C,showstringspaces=false]
int compteur;
for (compteur = 0 ; compteur < 10 ; compteur++)
{
    printf("Bienvenue sur OpenClassrooms !\n");
}
\end{lstlisting}

\end{tcolorbox}
\end{minipage}

\newpage
\noindent
\begin{minipage}[t]{0.5\textwidth}
\vspace{0pt}
\begin{tcolorbox}[title=Pointeurs, fonttitle=\large, top=0pt, bottom=0pt, boxsep=1pt, colback=white,
  colframe=tale, 
  colbacktitle=tale,
  coltitle=white,]
  \vspace{0.5em}
  \textbf{Pointeur} = variable qui contient l’adresse mémoire d’une autre variable
\vspace{-2em}
\subsection*{{\small \underline{Déclaration}}}
\vspace{-1.25em}
\noindent
\begin{minipage}[t]{0.5\textwidth} 
\begin{lstlisting}[language=C, basicstyle=\normalsize, aboveskip=0pt, belowskip=0pt]
int *p;
int x = 10;
int *p = &x;
\end{lstlisting}
\end{minipage}%
\hspace{4em}
\begin{minipage}[t]{0.45\textwidth}
\vspace{0.15em}
\textit{pointeur vers une entier}\\
\textit{p contient adresse de x}
\end{minipage}
\vspace{-1.5em}
\subsection*{{\small \underline{Afficher contenu d'un pointeur}}}
\noindent
\begin{lstlisting}[language=C, basicstyle=\normalsize, aboveskip=0pt, belowskip=20pt]
int age = 10;
int *pointeurSurAge = &age;
//Afficher adresse de age :
printf("%d", pointeurSurAge);
//Afficher valeur stockee dans adresse de age :
printf("%d", *pointeurSurAge);
\end{lstlisting}
\vspace{-1.5em}
  \end{tcolorbox}
  
  \begin{tcolorbox}[title=Tableaux, fonttitle=\large, top=5pt, bottom=0pt, boxsep=1pt, colback=white,
  colframe=blue!40!black!50, 
  colbacktitle=blue!40!black!50,
  coltitle=white]
  \subsection*{{\small \underline{Déclaration}}}
  \vspace{-1.25em}
  \noindent
\begin{minipage}[t]{0.5\textwidth} 
  \begin{lstlisting}[language=C, basicstyle=\normalsize, aboveskip=0pt, belowskip=0pt]
int notes[5];
int tab[3] = {1,2,3};
float moyennes[] = {12.5,14.0};
  \end{lstlisting}
  \end{minipage}%
\hspace{4em}
\begin{minipage}[t]{0.45\textwidth}
\vspace{0.15em}
\textit{déclaration simple}\\
\textit{déclaration et initialisation}\\
\textit{taille non précisée}
\end{minipage}
  \vspace{-1.25em}
\subsection*{{\small \underline{Acccès aux éléments}}}
\vspace{-1.25em}
\noindent
\begin{minipage}[t]{0.5\textwidth} 
\begin{lstlisting}[language=C, basicstyle=\normalsize, aboveskip=0pt, belowskip=0pt]
notes[0] = 15;
printf("%d",notes[0]);
\end{lstlisting}
\end{minipage}%
\hspace{4em}
\begin{minipage}[t]{0.45\textwidth}
\vspace{0.15em}
\textit{les indices commencent à 0}
\end{minipage}
\vspace{-1.25em}
\subsection*{{\small \underline{Tableau et pointeurs}}}
- Un tableau agit comme un pointeur constant vers son premier élément (tab[i] $\Leftrightarrow$ *(tab + i))\\
\noindent
\begin{minipage}[t]{0.5\textwidth} 
\begin{lstlisting}[language=C, basicstyle=\normalsize, aboveskip=-4pt, belowskip=0pt]
int *p = tab;
\end{lstlisting}
\end{minipage}%
\hspace{4em}
\begin{minipage}[t]{0.45\textwidth}
\vspace{-2pt}
\textit{équivaut à }p = \& tab[0]
\end{minipage}

  \end{tcolorbox}
  \begin{tcolorbox}[title=Matrices, fonttitle=\large, top=5pt, bottom=0pt, boxsep=1pt, colback=white,
  colframe=blue!55!cyan, 
  colbacktitle=blue!55!cyan,
  coltitle=white,]
  \subsection*{{\small \underline{Déclaration}}}
  \vspace{-1.25em}
  \noindent
\begin{minipage}[t]{0.5\textwidth} 
  \begin{lstlisting}[language=C, basicstyle=\normalsize, aboveskip=0pt, belowskip=0pt]
int matrice[2][3] = {
    {1, 2, 3},
    {4, 5, 6}
};
  \end{lstlisting}
  \end{minipage}%
\hfill
\begin{minipage}[t]{0.45\textwidth}
\vspace{0.15em}
\textit{2 lignes et 3 colonnes}
\end{minipage}
  \vspace{-1em}
\subsection*{{\small \underline{Accès aux éléments}}}
\vspace{-1.25em}
\noindent
\begin{minipage}[t]{0.5\textwidth} 
\begin{lstlisting}[language=C, basicstyle=\normalsize, aboveskip=0pt, belowskip=0pt]
matrice[i][j] = 10;
int x = matrice[1][2];
\end{lstlisting}
\end{minipage}%
\hfill
\begin{minipage}[t]{0.45\textwidth}
\vspace{0.15em}
\textit{modifie élément}\\
 \textit{récupère élément}
\end{minipage}
  \end{tcolorbox}
  \begin{tcolorbox}[title=Manipulation de fichiers, fonttitle=\large, top=5.5pt, bottom=5pt, boxsep=1pt, colback=white,
  colframe=orange!50!black!40, 
  colbacktitle=orange!50!black!40,
  coltitle=white,]

  - Utilisation de la bibliothèque standard <stdio.h>
  \vspace{-1.25em}
\subsection*{{\small \underline{Ouverture et fermeture d'un fichier}}}
\vspace{-1.25em}
\noindent
\begin{minipage}[t]{0.5\textwidth} 
\begin{lstlisting}[language=C, basicstyle=\normalsize, aboveskip=0pt, belowskip=0pt,showstringspaces=false]
FILE *fichier;
fichier=fopen("nom_du_fichier.txt","mode");
fclose(fichier);
\end{lstlisting}
\end{minipage}%
\hfill
\begin{minipage}[t]{0.45\textwidth}
\vspace{0.15em}
\textit{pointeur vers fichier}
\end{minipage}
  \vspace{-0.75em}
\subsection*{{\small \underline{Modes}}}
\begin{tabular}{|l|l|p{7.77cm}|}
\hline
\textbf{Mode} & \textbf{Signification} & \textbf{Condition}\\
\hline
"r" & Lecture seule & Le fichier doit exister\\
\hline
"w" & Écriture seule & Écrase le contenu s’il existe, sinon crée\\
\hline
"a"	& Ajout (append) & Ajoute à la fin du fichier, sinon crée\\
\hline
"r+" & Lecture/écriture	& Le fichier doit exister\\
\hline
"w+"	 & Lecture/écriture &	Écrase ou crée\\
\hline
"a+" & 	Lecture/écriture & 	Ajoute à la fin, crée si inexistant\\
\hline
\end{tabular}
\vspace{-1em}
\subsection*{{\small\underline{Lecture et écriture}}}
\begin{tabular}{|c|l|}
\hline
Écriture &
\begin{minipage}[t]{0.4\textwidth}
\vspace{-0.5em} 
\begin{lstlisting}[language=C, basicstyle=\normalsize, aboveskip=0pt, belowskip=0pt,showstringspaces=false]
fprintf(f,"A:%d,i:%c\n",a,i);
fputs("...",f);
fputc('A', f);
\end{lstlisting}
\end{minipage}%
\hspace{6em}
\begin{minipage}[t]{0.23\textwidth}
\vspace{-0.25em}
\textit{chaîne formatée}\\
\textit{chaîne simple}\\
\textit{caractère}
\end{minipage}\\[0.25em]
\hline
Lecture & \begin{minipage}[t]{0.4\textwidth}
\vspace{-0.5em} 
\begin{lstlisting}[language=C, basicstyle=\normalsize, aboveskip=0pt, belowskip=0pt,showstringspaces=false]
fscanf(f,"%s %d", nom, &age);
fgets(ligne,100,f);
char c = fgetc(f);
\end{lstlisting}
\end{minipage}%
\hspace{6em}
\begin{minipage}[t]{0.23\textwidth}
\vspace{-0.25em}
\textit{chaîne formatée}\\
\textit{ligne entière}\\
\textit{caractère}
\end{minipage}\\
\hline
\end{tabular}
\vspace{-1.25em}
\subsection*{{\small\underline{Fonctions utiles}}}
\begin{tabular}{|l|p{4cm}|}
\hline
\begin{lstlisting}[language=C, basicstyle=\normalsize, aboveskip=0pt, belowskip=0pt,showstringspaces=false]
remove("fichier.txt")
\end{lstlisting}
& Supprime un fichier
\\[0.25em]
\hline
\begin{lstlisting}[language=C, basicstyle=\normalsize, aboveskip=0pt, belowskip=0pt,showstringspaces=false]
rename("ancien.txt","nouveau.txt")
\end{lstlisting}
& Renomme un fichier
\\[0.25em]
\hline
\begin{lstlisting}[language=C, basicstyle=\normalsize, aboveskip=0pt, belowskip=0pt,showstringspaces=false]
fseek(f, position, origine)
\end{lstlisting}
& Déplace curseur
\\[0.25em]
\hline
\begin{lstlisting}[language=C, basicstyle=\normalsize, aboveskip=0pt, belowskip=0pt,showstringspaces=false]
ftell(f)
\end{lstlisting}
& Donne $(x;y)$ curseur
\\[0.25em]
\hline
\end{tabular}
  \end{tcolorbox}
  
  
  \end{minipage}
  \hfill
  \begin{minipage}[t]{0.49\textwidth}
\vspace{0pt}
  \begin{tcolorbox}[title=Structures, fonttitle=\large, top=5pt, bottom=0pt, boxsep=1pt, colback=white,
  colframe=blue!45!black, 
  colbacktitle=blue!45!black,
  coltitle=white,]
  \subsection*{{\small \underline{Déclaration}}}
  \vspace{-0.5em}
  \noindent
  \begin{lstlisting}[language=C, basicstyle=\normalsize, aboveskip=0pt, belowskip=0pt]
struct Etudiant {
    char nom[50];
    int age;
    float moyenne;
};
  \end{lstlisting}
  \vspace{-1em}
\subsection*{{\small \underline{Initialisation}}}
\vspace{-1.25em}
\noindent
\begin{minipage}[t]{0.5\textwidth} 
\begin{lstlisting}[language=C, basicstyle=\normalsize, aboveskip=0pt, belowskip=0pt]
struct Etudiant e1;
struct Etudiant e2 = {"Alice",20,14.5};
\end{lstlisting}
\end{minipage}%
\hspace{9em}
\begin{minipage}[t]{0.45\textwidth}
\vspace{0.15em}
 \textit{déclaration}\\
 \textit{initialisation}
\end{minipage}
\vspace{-1em}
\subsection*{{\small \underline{Accès aux éléments}}}
\vspace{-1.25em}
\begin{minipage}[t]{0.5\textwidth} 
\begin{lstlisting}[language=C, basicstyle=\normalsize, aboveskip=0pt, belowskip=0pt]
struct Etudiant *ptr = &e1;
printf("%f", ptr->moyenne);
printf("%s", e2.nom);
e1.age = 21;
\end{lstlisting}
\end{minipage}%
\hspace{6em}
\begin{minipage}[t]{0.45\textwidth}
\vspace{0.15em}
 \textit{ $->$ pour pointeur}\\
 \textit{. pour variable}
\end{minipage}
\vspace{-1em}
\subsection*{{\small \underline{Tableau de structures}}}
\vspace{-1.25em}
\begin{minipage}[t]{0.5\textwidth} 
\begin{lstlisting}[language=C, showstringspaces=false, basicstyle=\normalsize, aboveskip=0pt, belowskip=0pt]
struct Etudiant tab[3] = {
    {"Alice", 20, 14.5},
    {"Bob", 22, 13.2},
    {"Clara", 19, 15.8}
};
printf("%s : %.2f\n", tab[3].nom, tab[3].moyenne);
\end{lstlisting}
\end{minipage}%
\hspace{6em}
\begin{minipage}[t]{0.45\textwidth}
\vspace{0.15em}
 \textit{renvoie "Bob : 15.8"}
\end{minipage}
  \end{tcolorbox}
  \begin{tcolorbox}[title=Chaînes de caractères, fonttitle=\large, top=5pt, bottom=0pt, boxsep=1pt, colback=white,
  colframe= orange!30!pink, 
  colbacktitle=orange!30!pink,
  coltitle=white,]
  
\textbf{Chaîne de caractères} = tableau de char terminé par le caractère \verb|\0|
\vspace{-1.5em}
\subsection*{{\small \underline{Déclaration et initialisation}}}
\vspace{-1.25em}
\begin{minipage}[t]{0.5\textwidth} 
\begin{lstlisting}[language=C, basicstyle=\normalsize, aboveskip=0pt, belowskip=0pt]
char mot[10];
char nom[] = "Alice";
char prenom[20] = "Bob";
\end{lstlisting}
\end{minipage}%
\hspace{1em}
\begin{minipage}[t]{0.45\textwidth}
\vspace{0.15em}
\textit{chaîne vide}\\
\textit{taille déduite automatiquement}\\
\textit{taille fixe}
\end{minipage}
\vspace{-0.5em}
\subsection*{{\small \underline{Fonctions utiles}}}
\begin{tabular}{|l|p{8cm}|}
\hline
\textbf{Fonction} & \textbf{Description} \\
\hline
\begin{minipage}[t]{1.5cm}
\vspace{-\baselineskip}
\begin{lstlisting}[language=C, basicstyle=\small , aboveskip=5pt, belowskip=5pt]
strlen
\end{lstlisting}
\vspace{-\baselineskip}
\end{minipage}
& Renvoie la longueur d’une chaîne sans \verb|\0|\\
\hline
\begin{minipage}[t]{1.5cm}
\vspace{-\baselineskip}
\begin{lstlisting}[language=C, basicstyle=\small , aboveskip=5pt, belowskip=5pt]
strcpy
\end{lstlisting}
\vspace{-\baselineskip}
\end{minipage}
& Copie une chaîne dans une autre\\
\hline
\begin{minipage}[t]{1.5cm}
\vspace{-\baselineskip}
\begin{lstlisting}[language=C, basicstyle=\small , aboveskip=5pt, belowskip=5pt]
strncpy
\end{lstlisting}
\vspace{-\baselineskip}
\end{minipage}
& Copie jusqu’à n caractères\\
\hline
\begin{minipage}[t]{1.5cm}
\vspace{-\baselineskip}
\begin{lstlisting}[language=C, basicstyle=\small , aboveskip=5pt, belowskip=5pt]
strcmp
\end{lstlisting}
\vspace{-\baselineskip}
\end{minipage}
& Compare deux chaînes (0 si égales)\\
\hline
\end{tabular}
\vspace{-0.3em}
\subsection*{{\small \underline{Chaînes et pointeurs}}}
\vspace{-0.3em}
- Les chaînes de caractères peuvent être manipulées via un pointeur mais ne doivent pas être modifiées via ces derniers

\begin{minipage}[t]{0.5\textwidth}
\begin{lstlisting}[language=C, basicstyle=\small , aboveskip=5pt, belowskip=5pt]
char *s = "Hello";
printf("%c\n", *(s+1));
\end{lstlisting}
\end{minipage}%
\hspace{1em}
\begin{minipage}[t]{0.45\textwidth}
\vspace{0.15em}
\textit{affiche le caractère 'e'}
\end{minipage}
  \end{tcolorbox}
  
  \begin{tcolorbox}[title=Booléens, fonttitle=\large, top=5pt, bottom=5pt, boxsep=1pt, colback=white,
  colframe=black, 
  colbacktitle=black,
  coltitle=white,]
  \subsection*{{\small\underline{Déclaration}}}
- Définis dans la bibliothèque <stdbool.h>

\begin{minipage}[t]{0.5\textwidth} 
  \begin{lstlisting}[language=C, basicstyle=\normalsize, aboveskip=-4pt, belowskip=0pt]
#include <stdbool.h>
_Bool b1;  
bool b2;  
  \end{lstlisting}
  \end{minipage}%
\hfill
\begin{minipage}[t]{0.45\textwidth}
\vspace{1.15em}
\textit{type natif C, valeur 0 ou 1}\\
\textit{type bool fourni par <stdbool.h>}
\end{minipage}
  \vspace{-0.75em}
  \subsection*{{\small \underline{Afficher booléen}}}
\vspace{-1em}
\noindent
\begin{minipage}[t]{0.5\textwidth} 
\begin{lstlisting}[language=C, basicstyle=\normalsize, aboveskip=0pt, belowskip=0pt,showstringspaces=false]
bool a = true;
printf("a=%d\n", a);
printf("a=%s\n", a ? "true":"false");
\end{lstlisting}
\end{minipage}%
\hspace{7.5em}
\begin{minipage}[t]{0.45\textwidth}
\vspace{1.3em}
\textit{affiche 1}\\
 \textit{affiche a=true}
\end{minipage}
\vspace{-1em}
\subsection*{{\small \underline{Conditions et boucles}}}
\vspace{-1.25em}
\noindent
\begin{minipage}[t]{0.5\textwidth} 
\begin{lstlisting}[language=C, basicstyle=\normalsize, aboveskip=0pt, belowskip=0pt,showstringspaces=false]
while(a) {
 if (a) {...
} else {...
}
...
}
\end{lstlisting}
\end{minipage}%
\hfill
\begin{minipage}[t]{0.45\textwidth}
\vspace{0.15em}
\textit{tant que a=true}\\
\textit{si a=true}
\end{minipage}
\vspace{-0.5em}
\subsection*{{\small \underline{Portes logiques}}}
\begin{tabular}{|l|p{4cm}|}
\hline
\textbf{Symbole} & \textbf{Porte}\\
\hline
\&\& & ET\\
\hline
|| & OU\\
\hline
! & NON\\
\hline
\^{} & OU exclusif\\
\hline

\end{tabular}
  \end{tcolorbox}
  
  \end{minipage}

\end{document}
